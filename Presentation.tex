\documentclass{beamer}

\usetheme{CambridgeUS} % Puedes elegir el tema que prefieras

\title{Introducción a LaTeX}
\author{Reynol Gomez Franco C121}
\date{\today}

\begin{document}
	
	\begin{frame}
		\titlepage
	\end{frame}
	
	\begin{frame}
		\frametitle{¿Qué es LaTeX?}
		\LaTeX{} es un sistema de composición de documentos creado por Leslie Lamport basado en el lenguaje de programación TeX. Es especialmente utilizado para crear documentos técnicos o académicos de alta calidad tipográfica.
		
		\textbf{Ventajas de utilizar LaTeX:}
		\begin{itemize}
			\item Resultados profesionales de alta calidad.
			\item Facilita el manejo de ecuaciones, fórmulas matemáticas y símbolos especiales.
			\item Separación clara entre contenido y formato.
			\item Gestión automatizada de referencias y bibliografía.
			\item Amplia comunidad de usuarios y recursos disponibles.
		\end{itemize}
	\end{frame}
	
	\begin{frame}
		\frametitle{Ejemplo de Código LaTeX}
		Ejemplo básico de cómo se ve el código LaTeX para crear una lista numerada y una ecuación matemática:
		
		\begin{enumerate}
			\item Primer elemento de la lista.
			\item Segundo elemento de la lista.
			\item Tercer elemento de la lista.
		\end{enumerate}
		
		\begin{equation}
			e^{i\pi} + 1 = 0
		\end{equation}
		
		Este es solo un ejemplo simple, LaTeX ofrece muchas más funcionalidades y opciones de personalización.
		
	\end{frame}
	
	\begin{frame}
		
	\frametitle{Conclusión}
	LaTeX es una poderosa herramienta para crear documentos técnicos y académicos, ofreciendo resultados profesionales y una amplia gama de funcionalidades para manejar texto, ecuaciones y referencias. Si te encuentras en un entorno donde se valora la calidad tipográfica y la estructura formal de tus documentos, LaTeX puede ser una excelente opción para ti.
	
	¡Espero que esta breve introducción te haya dado una idea de lo que LaTeX puede ofrecer! ¡No dudes en explorar más y sumergirte en el maravilloso mundo de LaTeX!
	
\end{frame}

\end{document}